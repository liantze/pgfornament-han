%\documentclass[10pt,letterpaper]{beamer}
\documentclass[space,punct=kaiming]{ctexbeamer}
\usetheme{Xiaoshan}

%===================Theme
%\usetheme{Warsaw}
%\usetheme{Boadilla}
%\usetheme{Rochester}
\setbeamercovered{transparent} % Makes covered items partly visible
%===================
\usepackage{ragged2e}
\justifying
\let\raggedright\justifying % 修改默认左对齐为分散对齐
\usepackage{minted}
\newcommand{\mtl}[1]{\mintinline{tex}|#1|}
\newcommand{\mt}[1]{\mint{tex}|#1|}
\usepackage{indentfirst}
\xeCJKsetup{CJKecglue = {\hskip 0.08em plus 0.02em minus 0.01em},
	xCJKecglue = {\hskip 0.08em plus 0.02em minus 0.01em}}
% ----- Font Set -----
\setCJKmainfont[AutoFakeSlant, BoldFont = FZHTK.TTF]{FZZDXK.TTF}
\setCJKmonofont[AutoFakeSlant, AutoFakeBold]{FZFSK.TTF}
%===================Title Elements
\title[Beamer简介]{使用Beamer制作幻灯片}
\subtitle{取材于Lambert E. Murray的介绍}
\author{\href{mailto:yjccjlu@gmail.com}{圡の人}}
\date[2018]{戊戌年\\ 丙辰月\\ 甲戌日}%{\today}
\institute[量大]{之江畔\\ {\small 完颜侃数}}

%===========Begin Document
\begin{document}
%===========Title Slide
\frame{\maketitle}
% Or you can use:
% \begin{frame}
% \titlepage
% \end{frame}
%===========Begin Frames

\begin{frame}[containsverbatim]{文档结构}
\begin{columns}
\column{.34\textwidth}
使用 \LaTeX{} 制作幻灯片, 推荐使用 \emph{Beamer}.

通常, 你的文档需要有个标题, 作者, 每一章幻灯片应该有个 \emph{frame} 环境.

因此, 你的的第一个幻灯片大概看起来是这个样子.

\column{.66\textwidth}
\begin{block}{这是一个例子}
\begin{minted}{tex}
\documentclass[options]{beamer}
\usetheme{Boadilla}
\setbeamercovered{transparent}
\title{Long Title}
\subtitle{Short Title}
\author{Your Name}
\institute[HU]{Harding University}
\begin{document}
\begin{frame}
. . . 
\end{frame}
\end{document}
\end{minted}
\end{block}
\end{columns}
\end{frame}
 
\begin{frame}[fragile]{第一张幻灯片}
  这是第一张幻灯片的, 使用了默认的位置选项.
  \mt{\begin{frame}[option]}
  option: [t] - top; [b] - bottom; [c] - center (默认选项) \pause
 \begin{block}{块标题}
   幻灯片里的文本可以放在附带标题的块环境中.
   \mt{\begin{block}{块标题}}
 \end{block}\pause
 \begin{block}{}
   如果不想要块标题, 可以置空.
   \mt{\begin{block}{}}
  \end{block}
\end{frame}

\begin{frame}[t]{第二张幻灯片}
  这是第二张幻灯片.
\begin{block}{}
  这里使用了[t]选项使文本位于幻灯片顶部.
  \mt{\begin{frame}[t]{第二张幻灯片}}
\end{block}
\end{frame}

\begin{frame}[fragile]{第三张幻灯片}
  在幻灯片里可以使用典型的 \LaTeX{} 指令输入公式, 例如
  \begin{equation}
    \vec{F} = m \vec{a}.
  \end{equation}\pause
  
  还可以使用列表.

  \begin{itemize}
    \item 首先
    \item 其次
    \item 然后
  \end{itemize}\pause
  也可以使用有序列表.
  \begin{enumerate}
    \item 首先
    \item 其次
    \item 然后
  \end{enumerate}
\end{frame}

%============Columns
\begin{frame}[t,fragile]{多栏}
  可以将文本或公式放在不同的栏.
  
  这里我们设置两栏, 每一栏设为幻灯片文本宽度的一半.\pause
  
  \begin{columns}[t]
   \column{.45\textwidth}
    \begin{center}
      第一栏
    \end{center}\pause 
    内容可以是简单的文本, 
    
    也可以使用块环境.\pause 
    
   \column{.45\textwidth}
    \begin{center}
      第二栏
    \end{center}
    \begin{block}{公式}
      \[
        \vec{F}=m\vec{a}
      \]
    \end{block}\pause
    \begin{block}{文本}
      这是一句话.
    \end{block}
%    \begin{block}{}
%       一个简单的没标题的块.
%    \end{block}
  \end{columns}

\end{frame}
%==============Overlays
\begin{frame}[fragile]{覆盖和分项列表}
  常见的分项列表信息, 诸如
\begin{itemize}
\item 第一条
\item 第二条
\item 第三条
\end{itemize}\pause

可以使用\emph{overlays}每次展示一项来强化效果。 \pause

这实际上是由编译器创建独立的重复帧,

根据覆盖规范添加额外的项。
\end{frame}
%==============Overlays
\begin{frame}[fragile]{覆盖和分项列表}
\begin{columns}
\column{.45\textwidth}
\begin{block}{代码}
\begin{minted}{tex} 
\begin{itemize}
  \item<1-> 第一条
  \item<2-> 第二条
  \item<3-> 第三条
\end{itemize}
\end{minted}
\end{block}
\column{.45\textwidth}
\begin{block}{效果}
\	
\begin{itemize}
  \item<1-> 第一条
  \item<2-> 第二条
  \item<3-> 第三条
\end{itemize}
\
\end{block}
\end{columns}
\end{frame}
%============Variations on the Overlays
\begin{frame}[t,fragile]{列表项显示次序设定}
  可以通过使用叠加命令来改变显示顺序及特殊格式。
\begin{columns}
\column{.65\textwidth}
\begin{block}{代码}
\begin{minted}{tex} 
\begin{itemize}
  \item<3> 第一条
  \item<2-> \textbf<3>{第二条}
  \item<1> 第三条
\end{itemize}
\end{minted}
\end{block}
\column{.35\textwidth}
\begin{block}{效果}
 \	
\begin{itemize}
  \item<3> 第一条
  \item<2-> \textbf<3>{第二条}
  \item<1> 第三条
\end{itemize}
 \
\end{block}
\end{columns}
\end{frame}
%=================Using Uncover 
\begin{frame}[fragile]{关联指令uncover示例}
  列表标签有如下选择: [a], [A], [i], [I], [1]  \pause
\begin{columns}
\column{.65\textwidth}
\begin{block}{代码}
\begin{minted}{tex} 
\begin{enumerate}[1]
  \item<2-5>苹果
  \item<3-5>
  \color<5>[rgb]{0,0.6,0}猕猴桃
  \item<4-5>柠檬
\end{enumerate}
\uncover<1-5>{提示:}\\
\uncover<2-5>{苹果是圆的}\\
\uncover<3-5>{猕猴桃是毛绒绒的}\\
\uncover<4-5>{柠檬是酸的} 
\end{minted}
\end{block}
\column{.35\textwidth}
\begin{block}{效果}
\	
\begin{enumerate}[1]
	\item<2-5>苹果
	\item<3-5>
	\color<5>[rgb]{0,0.6,0}猕猴桃
	\item<4-5>柠檬
\end{enumerate}
\uncover<1-5>{提示:}\\
\uncover<2-5>{苹果是圆的}\\
\uncover<3-5>{猕猴桃是毛绒绒的}\\
\uncover<4-5>{柠檬是酸的} 
\
\end{block}
\end{columns}
\end{frame}
%==================Text Samples
\begin{frame}[fragile]{文本示例}

\begin{block}{\LaTeX{}指令\mtl{\uncover} \uncover<2>{及其效果}}
\begin{tabular}{l|c}
\hline
\multicolumn{1}{c}{\textbf{指令}}		&   \textbf{效果}\\ 
\hline
\verb|\textbf<2>{示例Sample}|		&	\textbf<2>{示例Sample}\\
\verb|\textit<2>{示例Sample}|		&	\textit<2>{示例Sample}\\
\verb|\textsl<2>{示例Sample}|		&	\textsl<2>{示例Sample}\\
\verb|\alert<2>{示例Sample}|		&	\alert<2>{示例Sample}\\
\verb|\textrm<2>{示例Sample}|		&	\textrm<2>{示例Sample}\\
\verb|\color<2>{green} 示例Sample|	&	\color<2>{green} 示例Sample\\
\end{tabular}
\end{block}



\end{frame}
\end{document}