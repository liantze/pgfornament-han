% !TEX program=xelatex
\documentclass[aspectratio=1610,linespread=1.2]{ctexbeamer}
\usetheme{TianQing}
% 可以把zhkai更改成自己喜欢的美术字体、书法字体等
% \setCJKfamilyfont{zhkai}{Xingkai SC}

\title{天青色等烟雨}
\author{林莲枝}
\subtitle{\texttt{pgfornament-han}附录福利}

% \setbeamercolor{structure}{fg=墨色}
% \setbeamercolor{section page deco}{fg=紫檀}
% \setbeamercolor{top deco}{fg=老银}
% \setbeamercolor{normal text}{bg=铅白}
% \setbeamercolor{alerted text}{fg=栗色}
% \setbeamercolor{example text}{fg=苍青}
% \setbeamercolor{block title}{fg=靛青}
% \setlength{\TQTopDecoWidth}{0.7\paperwidth}
% \setlength{\TQBottomDecoWidth}{0.2\paperwidth}
% \renewcommand{\TQTopDecoOpacity}{0.3}
% \renewcommand{\TQBottomDecoOpacity}{0.45}

\begin{document}

\begin{frame}
\maketitle
\end{frame}

\begin{frame}{\contentsname}
\tableofcontents
\end{frame}


\section{基本测试}

\begin{frame}{天青色等烟雨,而我在等你}
    \begin{itemize}
        \item 炊烟袅袅升起, 隔江千万里。
        \begin{itemize}
	        \item 在瓶底书刻隶仿前朝的飘逸
	        \begin{itemize}
		        \item 就当我为遇见你伏笔
	        \end{itemize}
        \end{itemize}
    \end{itemize}

\begin{enumerate}
        \item 本来这个beamer主题样式,想取名“青花瓷”的。不过始终没能力重现出来那种感觉啦,就算了。
        \item 话说拿这个模板去做科研学术性报告,真的不会被导师丢出来吗。
\end{enumerate}
    
\end{frame}

\begin{frame}
\frametitle{雨纷纷 旧故里草木深}
    \begin{enumerate}
        \item 我听闻 你始终一个人
        \begin{enumerate}
	        \item 斑驳的城门 盘踞着老树根
	        \begin{enumerate}
		        \item 石板上回荡的是 再等
		        \item 石板上回荡的是 再等
	        \end{enumerate}
        \end{enumerate}
    \end{enumerate}

\end{frame}

\begin{frame}[allowframebreaks]{各种block 123}
    \begin{exampleblock}{算了我也不知道在写什么,do you?}
    Now solve $x = \frac{-b \pm \sqrt{b^2 -4ac}}{2a}$. 对各位同学来说应该挑战不大。
    \end{exampleblock}
    
    \begin{alertblock}{算了我也不知道在写什么,do you?}
    \[ x = \frac{-b \pm \sqrt{b^2 -4ac}}{2a} \]
    \end{alertblock}
    
    \begin{block}{算了我也不知道在写什么,do you?}
    \[ x = \frac{-b \pm \sqrt{b^2 -4ac}}{2a} \]
    \end{block}
    
    \begin{proof}
    显而易见,$1+1=2$.
    \end{proof}

    \begin{theorem}
    有一件很美好的事情将要发生,它终会发生。
    \end{theorem}
\end{frame}

\section{内容不要太长,写短一点,汇报是你讲不是观众读片}
    
\end{document}
